\documentclass{article}
\usepackage{amsmath}
\usepackage{mathrsfs}
\usepackage{enumitem}
\usepackage[margin=3cm]{geometry}
\usepackage{titlesec}
\author{Alex Seewald}
%            type        commands to apply to text  spacing
\titleformat{\chapter}[hang]{\Huge\center}{}{\smallskip}{}{}
\titleformat{\section}[hang]{\huge\center}{}{\smallskip}{}{}
\titleformat{\subsection}[hang]{\Large}{}{\smallskip}{}{}
\titleformat{\subsubsection}[hang]{\large}{}{\smallskip}{}{}
\title{Physics / Mathematics Review}
\begin{document}

\newlength{\lindent}
\setlength{\lindent}{1.3cm}

\maketitle

\section*{Classical Physics}

    \subsection*{Rotational Motion}
        \begin{align*}
             I &= \sum m_i r_i^2 &  K &= \frac{1}{2} I \omega^2 & \tau &= r \times F  & \tau &= I \alpha
        \end{align*}

    \subsection*{Gravitation}
        $ F = G \frac{m_1 m_2}{r^2} $
        \begin{itemize}[leftmargin=\lindent]
        \item Shell Theorem : A uniform shell of matter exerts no net gravitational
        force on a particle located inside it.
        \item Things are heavier on the equator because, in addition to $m a_g$ down, there
        is also the centripetal force of the earth turning on its axis.
        \end{itemize}

    \subsection*{Waves}
        \begin {align*}
            d \sin(\theta) &= m \lambda & f' &= f \frac{v \pm v_D}{v \pm v_S}
        \end{align*}
        \begin{itemize}[leftmargin=\lindent]
        \item Closed end pipe : standing waves occur when the wave hits the closed end at the middle (so
        there are an odd number of half periods (wave numbers)).
        \item Open end pipe : standing waves occur when the wave hits the open ends at the edge (so there
        are an even number of half periods (wave numbers)).
        \end{itemize}
    \subsection*{Heat}
        \begin {align*}
        \Delta Q &= cm \Delta T \hspace{2em} \text{where c is specific heat.} &
        \end{align*}
        \begin{itemize}[leftmargin=\lindent]
        \item Two definitions for entropy: $S = \int_i^f \frac{dQ}{T}$, $S = k ln( \frac{N!}{n_1! n_2!} )$
        \item Ideal Gas Law: $P V = n \kappa T$
        \item Equipartition Theorem:
        \end{itemize}
    \subsection*{E \& M}
        \begin{align*}
             V_f - V_i &= - \int_i^f \vec{E} \cdot d\vec{s} & \mathscr{E} &= - \frac{d\Phi_B}{dt}
        \end{align*}
        \begin{itemize}[leftmargin=\lindent]
        \item Like with Gravity's shell theorem, a charged particle inside a shell of uniform charge has no
        net electrostatic force on it. Also, like with gravity's shell theorem, a charged particle outside
        the shell is attracted or repelled as if all the charge is at the center.
        \item Potential due to a point charge: $V = \frac{1}{4 \pi \epsilon_0} \frac{q}{r}$
        \item The electrical potential energy of a system of fixed point charges is equal to the work
        that must be done by an external agent to assemble the system, bringing in each charge
        from an infinite distance.
        \item An induced current has a direction such that the magnetic field due to the current opposes
        the change in the magnetic flux that induces the current.
        \item Definition of inductance for a solenoid with N turns: L = $\frac{N\Phi_B}{i} $
        \item Self-induced emfs: $\mathscr{E}_L = -L \frac{di}{dt}$
        \item Impedance is a generalization of resistance that applies to RLC circuits.
        \item Magnetic Dipole moment (of a coil): $\mu  = N i A \vec{n}$. So, it is normal to the coil's circle.
              $U = - \mu \cdot B$, $\tau = \mu X B$. where the torque attempts to align B and $\mu$ and the
              energy of the field is `orientation energy' (potential energy stored in orientation). 
        \item Electrical Dipole. The `moment' vector, $\vec{p}$ points from the positive to negative point 
        charge of the dipole. $\vec{\tau} = \vec{p} X \vec{E}$.

        \end{itemize}
        \subsubsection*{Maxwell's Equations}
        \begin{enumerate}[leftmargin=\lindent]
            \item $\oint \vec{B} \cdot \vec{A} = \Phi_B = 0$ : a mathematical statement that no magnetic monopoles exist.
            \item $\oint \vec{E} \cdot \vec{A} = \frac{q_enc}{\epsilon_0}$
            \item $\oint \vec{E} \cdot \vec{s} = - \frac{\Phi_B}{dt}$
            \item $\oint \vec{B} \cdot \vec{s} = \mu_0\epsilon_0 \frac{\Phi_E}{dt} + \mu_0 i_enc$
        \end{enumerate}
        What is the electrical flux through? The area enclosed by the current loop.
        
    \subsection*{Circuits}
        $q = C V$ is a property of all capacitors. The definition of a system's capicitance
        depends on the shape of the capacitor.
        Conventions: electric field lines point from + to - regions (so from one capacitor plate to another is
        an example).
        \subsubsection*{Loop Rules}
        \begin{enumerate}
            \item The algebraic sum of the changes in potential encountered in a complete traversal
            of any loop of a circuit must be zero.
            \item The sum of the currents entering any junction must be equal to the sum of the currents
            exiting the junction.
        \end{enumerate}
        For capacitors in parallel: $ C_eq = \sum\limits_{i=0}^{n} C_i$
        For capacitors in series: $ \frac{1}{C_eq} = \sum \frac{1}{C_i}$
        Energy stored in electric field of a capacitor: $ U = \frac{q^2}{2C}$
        Energy stored in the magnetic field of an inductor: $U = \frac{L i^2}{2}$
        Dialetrics, physically manifested by stuff between the capacitor plates, cause any instance of $\epsilon_0$ in an equation
        to $ \epsilon_0 \kappa $.
        Definition of Resistance: $ R = \rho \frac{L}{A}$ where $\rho$, resistivity, is a property of the material and L is length.
        Ohm's law is an approximation that holds well under low-heat conditions.
        $q_enc = \epsilon_0 \oint \vec{E} \cdot d\vec{A} $ which is a surface integral.
        \subsubsection*{RC Circuits}
        Charge builds up on the capacitor exponentially and decays with the 'sawtooth' pattern.
        \subsubsection*{RL Circuits}
        When a switch is turned on, rather than the current across R approaching $\frac{\mathscr{E}}{R}$, the
        an opposing emf $-L \frac{i}{dt}$ is induced. As the current stops increasing, the circuit
        starts to act like an ordinary R circuit. The current expression involves solving the differential eqn.
        \subsubsection*{LC Circuits}
        Whereas RL and RC circuits' current grows and decays exponentially, LC circuits' current varies sinusoidally.
        Like a block-spring system, there is a notion of driving frequency: $\omega = \frac{1}{LC}$
        \subsubsection*{RLC Circuits}
        RLC circuits are like LC circuits but damped. Rather than $\frac{dU}{dt} = 0$, $\frac{dU}{dt} = - i^2 R $
        \begin{itemize}[leftmargin=\lindent]
        \item If an external emf is added, damping does not occur. Most AC circuits are RLC.
        \item Reactance (X) - the factor $V = X I$. Resistive, Capacitave, and Inductive Reactance exist.
        \item $X_C = \frac{1}{\omega_d C}$
        \item $X_L = \omega_d L$.
        \item $Z = \sqrt{R^2 + (X_L - X_C)^2}$. Makes sense because $X_L$ and $- X_C$ are colinear and $R$ is
        orthogonal to that. 
        \item $<whatever>_RMS = \frac{<whatver>_max}{\sqrt{2}}$, provied that <whatever> varies sinusoidally.
        \item Transformers: 
        \end{itemize}

    \subsection*{Special Relativity}
        \begin {align*}
        \gamma &= \frac{1}{\sqrt{1 - (v/c)^2}} & \delta t &= \gamma \delta t_0 & L &= \frac{L_0}{\gamma}
        E &= \gamma m c^2 &= m c^2 + K & \vec{p} &= \gamma m \vec{v}
        \end{align*}
        \begin{itemize}[leftmargin=\lindent]
        \item Simultinaity is relative.
        \item When chemical reactions occur, the Q of the reaction is equal to the mass lost times $c^2$.

        \end{itemize}
    \subsection*{Optics}
    \begin{itemize}[leftmargin=\lindent]
        \item Plane mirrors can be thought of as spherical mirrors with an infinitely long radius of curvature.
        \item For plane mirrors, i = -p.
        \item In a concave spherical mirror, incidently parallel light rays intersect at a real focus.
        \item In a convex sphereical mirror, incidently parallel light rays intersect at a virtual focus.
        \item For a spherical mirror, f = $\frac{r}{2}.$ where r is radius of curvature and f is focal distance.
              (this assumes that the incident rays are equidistant from radius of curvature).
        \item $ \frac{1}{p} + \frac{1}{i} = \frac{1}{f}$, so when p = f, the virtual image is infinitely far
        away. So, when p < f, i is negative. So, when p > f, i is positive. As p approaches infinity, i = f.
        \item magnification for spherical mirror: $m = - \frac{i}{p}.$
        \item With mirrors, the real image is on the object's side and with lenses, the virtual image is on
        the object's side.
        \item Thin lens formula: $\frac{1}{f} = (n - 1) ( \frac{1}{r_1} - \frac{1}{r_2} )$
        where $r_1$ is the radius of curvature of the lens half on the object's side and $r_2$ is that on the
        other side.
    \end{itemize}

    \subsection*{Fluids}
        $F_b = m_f g $ where $m_f $is the mass of the fluid displaced.

    \subsection*{Parameters and Constants}
        \begin{align*}
             G &= & \mu_0 &=  & e &= 1.602 X 10^-19 C & k &= 1.381 X 10^-23 J / K 
             \epsilon_0 &= 8.854 X 10^-12 F/m & \mu_0 = 1.257 X 10^-6 H/m & 
            h &= 6.63 X 10^-34 J \cdot s & 
        \end{align*}

\section*{Modern}
    %photoelectric effect.
    \subsection*{Planck's Constant}
    For blackbodies, $E = h f$ ($f = v / \lambda $ for waves in general).
    For light, $E = \frac{hc}{\lambda}$
    Definition of de Broglie wavelength: $\lambda = \frac{h}{p}$.
    $\vec{L} = \sqrt{l (l-1)} h$ and $\mu_orb = -\frac{e}{2m} \vec{L}$ (same units as $C/s m^2$).
    \subsection*{Photoelectric Effect}
    Metals emit electrons when light shines on them.
    $ K_max = h (f - f_0)$ where f is the frequency of incident frequency of photon and $f_0$ is the
    threshhold at and below which no electron is emitted. This threshold is a property of the material.
    \subsection*{Thompson Scattering}

    \subsection*{Uncertainty Principle}
    $ \delta x \delta p \geq \frac{h}{2 \pi}$.

    \subsection*{Pauli Exclusion Principle}
    No two particles may have all the same quantum numbers in the same atom. (applies to fermions, e.g. electrons).
    This explains why electrons are repulsed from one another and why neutrons can simply clump together.
    \subsection*{Quantum Numbers}
    Bosons have integer spin quantum numbers.
    Fermions have half integer spin quantum numbers.
    Fermions obey the pauli exclusion principle.
    Hadrons have the strong force acting on them.
    Leptons do not have the strong force acting on them.
    $Mesons ::= bosons \cup hadrons$, $baryons ::= fermions \cup hadrons$.
    \subsection*{What is the significance?}
        \begin{itemize}[leftmargin=\lindent]
            \item Elastic Collisions - In inelastic collisions, heat loss is a big enough factor
                  that energy is not said to be conserved.
            \item Degrees of Freedom
            \item Blackbody Radiation
            \item Spin (quantum)
            \item Antiparticles - all reversible quantum numbers are reversed.
            \item Phasors: A construct that represents a wave. It is a vector of length equal to the
            maximum magnitude of the wave. It rotates about the origin with an angular speed equal to
            the angular frequency of the wave. The projection of the rotating phasor onto the vertical
            axis is equal to the displacement of a point on the wave from the wave's axis of symmetry.
            Phasors are useful for thinking about how waves add (vector addition laws apply).
        \end{itemize}
    \subsection*{Analogies and Connections}
        \begin{itemize}[leftmargin=\lindent]
            \item Gauss' Law <=> Surface Integrals.
            \item Definition of Work <=> Line Integrals.
            \item Taylor Series <=> Motion with arbitrary rates of change.
        \end{itemize}

   \subsection*{Open Questions}
        \begin{description}
        \item[Physical Correspondence] How is it that capacitors are not like closed switches in a circuit?
        They do not make a physical connection between the sides of the wire.
        \end{description}

\section*{Trig Identities}
    \begin{align*}
        \sin^2{\theta} + \cos^2{\theta} &= 1                 & 1 + \tan^2{theta} &= \sec^2{theta} \\
        \sin{u \pm v} &= \sin{u} \cos{v} \pm \cos{u} \sin{v} & \cos{u \pm v} &= \cos{u} \cos{v} \mp \sin{u} \sin{v}
    \end{align*}

\section*{Unit Identities}
       Fundamental Units
       \begin{enumerate}
       \item meter
       \item second
       \item columb
       \item kilogram
       \end{enumerate}
    \begin{align*}
        c &= \frac{1}{\sqrt{\mu_0 \epsilon_0}}
        N &= kg \frac{m}{s^2} & J &= kg \frac{m^2}{s^2}
        E &= N / C & V &= E m
        T &= \frac{kg}{s C} & Wb &= T m^2 
    \end{align*}

\end{document}
